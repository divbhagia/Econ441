\documentclass{./../../Latex/homework}
\begin{document}
\thispagestyle{plain}
\myheader{Homework 7 Solutions}

%%%%%%%%%%%%%% Exercise 8.5
\subsection*{Exercise 8.5}

\begin{enumerate}

% Question 1
\item[1.] For each $F(x, y) = 0$, find $d y / d x$ for each of the following:
\begin{tasks}(1)
\task $y-6 x+7=0$ $$
\frac{d y}{d x}-6=0 \rightarrow \frac{d y}{d x}=6
$$

\task $3 y+12 x+17=0$
$$
 3 \frac{d y}{d x}+12=0 \rightarrow \frac{d y}{d x}=\frac{-12}{3}=-4
$$

\task $x^{2}+6 x-13-y=0$
$$
 2 x+6-\frac{d y}{d x}=0 \rightarrow \frac{d y}{d x}=2 x+6
$$
\end{tasks}

% Question 2
\item[2.] 
\begin{enumerate}
\item[(d)] $F(x, y)=6 x^{3}-3 y=0$. By implicit function theorem:
$$
\begin{aligned}
& \frac{d y}{d x}=\frac{-F_{x}}{F_{y}} \\
& F_{x}=18 x^{2}, \quad F_{y}=-3 \\
& \frac{d y}{d x}=\frac{-18 x^{2}}{-3}=6 x^{2}
\end{aligned}
$$

\end{enumerate}

% Question 3
\item[3.] 
\begin{enumerate}
\item[(a)] $F(x, y, z)=x^{2} y^{3}+z^{2}+x y z=0$. By implicit function theorem:
$$
\begin{aligned}
& \frac{\partial y}{\partial x}=\frac{-F_{x}}{F_{y}}=\frac{-\left(2 x y^{3}+y z\right)}{3 x^{2} y^{2}+x z} \\~\\
& \frac{\partial y}{\partial z}=\frac{-F_{z}}{F_{y}}=\frac{-(2 z+x y)}{3 x^{2} y^{2}+x z}
\end{aligned}
$$
\end{enumerate}

\end{enumerate}

%%%%%%%%%%%%%% Exercise 14.2
\subsection*{Exercise 14.2}

\begin{enumerate}	
% Question 1
\item[1.] Find the following: 
\begin{tasks}(1)
\task[(a)] $\displaystyle\int 16 x^{-3}\ d x= \frac{16x^{-2}}{-2}+c=-8 x^{-2}+c \quad(x \neq 0)$
\task[(c)] $\displaystyle\int (x^5-3x) \ dx =\int x^5 \ dx-3 \int x \ dx=\frac{x^6}{6}-\frac{3x^2}{2} +c$
\task[(d)] $ \displaystyle\int 2e^{-2x} \ dx = 2 \displaystyle\int e^{-2x} \ dx = 2 \frac{e^{-2x}}{-2} + c= -e^{-2x} + c$
\end{tasks}
\end{enumerate}
%%%%%%%%%%%%%% Exercise 14.3
\subsection*{Exercise 14.3}

\begin{enumerate}

% Question 1
\item[1.] Evaluate the following:
\begin{tasks}(1)
\task[(a)] $ \displaystyle\int_1^3 \dfrac{1}{2}x^2 \dx = \left[\frac{x^3}{6} \right]_1^3=\frac{3^3-1^3}{6}=\frac{26}{6}$
\task[(e)] $\displaystyle \left[\frac{ax^3}{3}+\frac{bx^2}{2}+c x\right]_{-1}^1=\left(\frac{a}{3}+\frac{b}{2}+c\right)-\left(-\frac{a}{3}+\frac{b}{2}-c\right)=2\left(\frac{a}{3}+c\right)$ 
\end{tasks}

% Question 2
\item[2.] Evaluate the following:
\begin{tasks}(1)
\task[(a)] $ \displaystyle\int_1^2 e^{-2x} \ dx = \left[\frac{-e^{-2 x}}{2} \right]_1^2=-\frac{1}{2}\left(e^{-4}-e^{-2}\right)=\frac{1}{2}\left(e^{-2}-e^{-4}\right) $
\task[(d)] $ [\ln x+\ln (1+x)]_e^6= \ln 6+ \ln 7-\ln e - \ln (1+e) = \ln 42-1-\ln(1+e)$
\end{tasks}

% Question 3
\item[5.] Verify that a constant $c$ can be equivalently expressed as a definite integral:
\begin{tasks}(1)
\task $ \displaystyle \int_0^b \dfrac{c}{b}\ dx = \left[\frac{cx}{b}\right]_0^b = \frac{cb}{b}-\frac{c.0}{b} = c	$
\task $ \displaystyle  \int_0^c 1\ dt = [t]_0^c = c-0 = c 	$
\end{tasks}

\end{enumerate}

\end{document}