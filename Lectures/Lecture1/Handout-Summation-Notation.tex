\documentclass{./../../Latex/handout}
\begin{document}
\thispagestyle{plain}
\myheader{Summation Notation}

The capital sigma ($\Sigma$) stands for summing everything on the right. 
$$ \sum_{i=1}^N x_i = x_1 + x_2 + ... + x_N $$
\underline{Things you CAN do to summations:}
\begin{enumerate}
\item Pull constants out of them, or into them.
$$ \sum_{i=1}^N b x_i = b \sum_{i=1}^N x_i  $$
\item Split apart (or combine) sums (addition) or differences (subtraction)
$$ \sum_{i=1}^N (b x_i + c y_i) = b \sum_{i=1}^N x_i  + c \sum_{i=1}^N y_i $$
\item Multiply through constants by the number of terms in the summation
$$ \sum_{i=1}^N (a+b x_i)= aN + b \sum_{i=1}^N x_i  $$
\end{enumerate}

\underline{Things you CANNOT do to summations:}
\begin{enumerate}
\item Split apart (or combine) products (multiplication) or quotients (division).
$$ \sum_{i=1}^N x_i y_i \neq  \sum_{i=1}^N x_i \times \sum_{i=1}^N y_i   $$
\item Move the exponent out of or into the summation.
$$ \sum_{i=1}^N x_i^a \neq  \left(\sum_{i=1}^N x_i\right)^a $$
\end{enumerate}

\underline{Exercise}:
$$ x = \{2,9,6,8,11,14\} \quad \quad y = \{7,1,3,5,0\}$$ 
\begin{enumerate}
\item $\mathlarger{\sum}_{i=1}^4 x_i = $ \\~\\
\item $\mathlarger{\sum}_{i=1}^4 2 x_i = $ \\~\\
\item $\mathlarger{\sum}_{i=1}^4 (x_i+4) = $ \\~\\
\item $\mathlarger{\sum}_{i=1}^3 (x_i+y_i) = $ \\~\\
\item $\mathlarger{\sum}_{i=1}^2 x_i y_i = $ \\~\\
\item $\mathlarger{\sum}_{i=1}^2 x_i \times \mathlarger{\sum}_{i=1}^2 y_i = $ \\~\\
\item $\mathlarger{\sum}_{i=1}^2 x_i^2 $ \\~\\
\end{enumerate}

\end{document}