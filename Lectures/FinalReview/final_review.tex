\documentclass{./../../Latex/teaching_slides}
\usepackage{venndiagram}
\usepackage{tikz}
\usepackage{caption}
\usepackage{subcaption}
\usepackage{pgfplots}
%\usepackage{siunitx}
\pgfplotsset{compat=newest}
\usetikzlibrary{arrows.meta}

\begin{document}

\title{ECON 441 \\ \vspace{0.4em} \normalsize Introduction to Mathematical Economics}
\author{Div Bhagia}
\date{Final Exam Review}

%%%%%%%%%%%%%%%%%%%%
\begin{frame}[noframenumbering, plain]
\maketitle
\end{frame}

%%%%%%%%%%%%%%%%%%%%
\begin{frame}{Numbers, Sets, and Functions}
\begin{witemize}
\item Types of numbers: integers, fractions, rational numbers, irrational numbers, real numbers.  
  \item Set notation: \\ \vspace{0.1em}
     Example: $ A = \{a, b, c, d\} \quad or \quad A = \{x | x \in  \mathbb{R}\} $
\item Set relations: equivalence, subset, disjoint
\item Set operations: union, intersection, complement
\end{witemize}
\end{frame}

%%%%%%%%%%%%%%%%%%%%
\begin{frame}{Numbers, Sets, and Functions}
\begin{witemize}
\item  Cartesian product \\ \vspace{0.1em} Example: $ \mathbb{R}^2 =  \{(x,y) | x \in \mathbb{R}, y \in \mathbb{R}\} $
\item Relation: subset of a Cartesian product
\item Function: a relation where for each $x$ there is a unique $y$
 $$ f: X \rightarrow Y, \quad y = f(x) $$ 
$X:$ domain, $Y:$ codomain, $f(X):$ range
\end{witemize}
\end{frame}

%%%%%%%%%%%%%%%%%%%%
\begin{frame}{Numbers, Sets, and Functions}
\begin{witemize}
 \item One-to-one function: each value of $y$ is also associated with a unique value of $x$
\item One-to-one mapping unique to strictly monotonic functions
\item Inverse of a function only exists for strictly monotonic functions $$x=f^{-1}(y)$$ returns the value corresponding value of $x$ for each $y$. 
\end{witemize}
\end{frame}



%%%%%%%%%%%%%%%%%%%%
\begin{frame}{Summation Notation}
Example 1:
$$ \sum_{i=1}^3 \sum_{j\leq i} X_i Y_j $$
\end{frame}

%%%%%%%%%%%%%%%%%%%%
\begin{frame}{Summation Notation}
Example 2:
$$ \sum_{i=1}^2 \sum_{j=1}^2 (X_i Y_j + 4Y_j^2 + 1)$$
\end{frame}

%%%%%%%%%%%%%%%%%%%%
\begin{frame}{Linear Algebra}
\vspace{-.5em}
\begin{witemize}
  \item Matrix operations: addition, subtraction, scalar multiplication, matrix multiplication
  \item Identity matrix, transpose of a matrix
  \item Inverse of a matrix: $ A A^{-1} = A^{-1} A  = I $
  \item Solution of a linear-equation system $Ax = b$
  $$ A^{-1} A x  =  A^{-1}b \rightarrow x = A^{-1}b  $$
  \item Finding the determinant $|A|$ and inverse of a matrix $$A^{-1} = \frac{1}{|A|}Adj A$$
\end{witemize}
\end{frame}

%%%%%%%%%%%%%%%%%%%%
\begin{frame}{Linear Algebra}
\begin{witemize}
  \item  If a matrix's inverse exists, it's called a \textbf{nonsingular} matrix
\item Necessary and sufficient conditions for nonsingularity: \vspace{0.25em}
\begin{itemize}
  \item \textit{Necessary}: square matrix
  \item \textit{Sufficient}: rows (or equivalently columns) are linearly independent
\end{itemize}
\item Rank of matrix: maximum number of linearly independent rows (square matrix with full rank = nonsingular)
\item For singular matrices the determinant $|A|=0$
\end{witemize}
\end{frame}

%%%%%%%%%%%%%%%%%%%%
\begin{frame}{Linear Algebra}
\vspace{-.5em}
Say we have the following system of equations:
$$
\begin{aligned}
& 3x+2y=20 \\
& 6 x + 4 y =40
\end{aligned}
$$

Can write this as:
$$
A v=b
$$
where
$$
A=\left[\begin{array}{cc}
3 & 2 \\
6 & 4
\end{array}\right] \quad v=\left[\begin{array}{l}
x \\
y
\end{array}\right] \quad b=\left[\begin{array}{c}
20 \\
40
\end{array}\right]
$$
Unique solution for this system does not exist as $A$ is singular.
\end{frame}

%%%%%%%%%%%%%%%%%%%%
\begin{frame}{Linear Algebra}
\vspace{-.5em}
Let's solve the following system of equations $$
\begin{aligned}
& 3x+2y=20 \\
& 6 x - 3 y =40
\end{aligned}
$$

\end{frame}

%%%%%%%%%%%%%%%%%%%%
\begin{frame}{Calculus}
\begin{witemize}
  \item Limit definition of differentiability and continuity
  \item Rules of differentiation to differentiate functions (including log and exponential functions)
  \item Partial and total derivatives
  \item Second-order derivatives
  \item Elasticities and partial elasticities 
\end{witemize}

\end{frame}


%%%%%%%%%%%%%%%%%%%%
\begin{frame}{Calculus}
For the function:
$$ y = f(x_1, x_2,...,x_n) $$ 
\vspace{1em}

Note that the gradient and Hessian is given by
$$ \nabla f = \begin{bmatrix}
	f_1 \\
	f_2 \\
	\vdots \\
	f_n
\end{bmatrix} \quad \quad
H = \begin{bmatrix}
	f_{11} & f_{12} & \hdots & f_{1n} \\
	f_{21} & f_{22} & \hdots & f_{2n} \\
	\vdots & \vdots & & \vdots \\
	f_{n1} & f_{n2} & \hdots & f_{nn} \\
\end{bmatrix} $$
\end{frame}

%%%%%%%%%%%%%%%%%%%%
\begin{frame}{Calculus}
Calculate the total differential of the production function $$Q = F(K,L)$$ to find how a small change in both labor and capital affects the production.
\end{frame}


%%%%%%%%%%%%%%%%%%%%
\begin{frame}{Calculus}
\vspace{-0.25em}
Consider a company that allocates its marketing budget \( X(a) \) based on the economic climate, represented by an economic index \( a \).
\[ X(a) = 10a \] \vspace{0.1em}

 The company's sales revenue \( Y \) depends on both the marketing spend \( X(a) \) and the economic index \( a \).

\[ Y(X(a), a) = X(a) \cdot \log(1 + a) \]  \vspace{0.1em}

How does this company's revenue vary with respect to the economic index $a$?
\end{frame}


%%%%%%%%%%%%%%%%%%%%%%%%%%%% Optimization

%%%%%%%%%%%%%%%%%%%%
\begin{frame}{Single-Variable Optimization}
\begin{witemize}
\item Given a function
$$ y = f(x) $$ 
  \item Critical point $f^{\prime}\left(x^*\right)=0$, necessary condition for an optimum
  \item Sufficient condition: \vspace{0.25em}
  \begin{itemize}
  \normalsize
  \item maximum if $f''(x^*)<0$
  \item minimum if $f''(x^*)>0$ \\~\\
\end{itemize}
\end{witemize}
\end{frame}

%%%%%%%%%%%%%%%%%%%%
\begin{frame}{Single-Variable Optimization}
Let's find the extrema for the following function and plot it:
$$ f(x) = x^4-2x^2 $$
\end{frame}

%%%%%%%%%%%%%%%%%%%%
\begin{frame}{Single-Variable Optimization}
Say, $f(x)$ is a strictly concave function and 
$$f'(x^*) = 0 $$ 
Is $f(x^*)$ the global maximum? Can you explain why?
\end{frame}

%%%%%%%%%%%%%%%%%%%%
\begin{frame}{Multiple-Variable Optimization}
$$ y = f(x_1, x_2,...,x_n) $$

First-order condition:
$$ \nabla f (x_1,x_2,...,x_n)  = 0 $$
That is:
$$ f_1(x_1,x_2,...,x_n) = 0$$
$$ f_2(x_1,x_2,...,x_n) = 0$$
$$ \vdots $$ 
$$ f_n(x_1,x_2,...,x_n) = 0$$

\end{frame}

%%%%%%%%%%%%%%%%%%%%
\begin{frame}{Multiple-Variable Optimization}
\vspace{-1.5em}
$$ \pi(K, L) = AK^{1/3}L^{2/3}-wL-rK  $$ 
Show that at the optimal: $wL = 2 rK$ \\
 \begin{tikzpicture}[remember picture,overlay]
    \draw[gray!50, opacity=0.75] (0,-5.25) grid[step=0.75] (14.25,0);
  \end{tikzpicture}
\end{frame}

%%%%%%%%%%%%%%%%%%%%
\begin{frame}{Envelope Theorem}
Value function:
$$
V=f\left(x^{*}(\alpha), y^{*}(\alpha), \alpha\right)
$$

\vspace{0.5em}
If we differentiate $V$ with respect to $\alpha$:
$$
\frac{d V}{d \alpha} =f_{x}^{*} \cdot \frac{d x^{*}}{d \alpha}+f_{y}^{*} \cdot \frac{d y^{*}}{d \alpha}+f_{\alpha}^{*}
$$

\vspace{0.5em}
From the first order conditions we know $f_{x}^{*}=f_{y}^{*}=0$, therefore
$$
\frac{d V}{d \alpha}=f_{\alpha}^{*}
$$
\end{frame}

%%%%%%%%%%%%%%%%%%%%
\begin{frame}{Envelope Theorem}

$$ V= \pi(K^*, L^*)$$ \\~\\

How does optimal profit change due to a change in $w$ or $r$?
\end{frame}

%%%%%%%%%%%%%%%%%%%%
\begin{frame}{Constrained Optimization}
\vspace{-1.25em}
$$ 	U = U(c_1, c_2) =  \ln c_1 + \beta \ln c_2 \quad \quad 0<\beta<1$$
\vspace{-0.65em}
\begin{witemize}
  \item $y_1,y_2>0$: income in period 1 and 2
  \item Income you save $s$ in period 1 earns interest $r>0$
  \item In which case,
 $$ c_1 + s = y_1 \quad \quad \quad  c_2 = y_2 + (1+r) s $$
 \item Combining these constraints:
$$ c_1 + \frac{1}{1+r} c_2 = \underbrace{y_1 + \frac{1}{1+r} y_2}_{m \equiv \text{present-discounted income}}$$
%$$ m = y_1 + \frac{1}{1+r} y_2 $$}} $$
\end{witemize}
\end{frame}

%%%%%%%%%%%%%%%%%%%%
\begin{frame}{Constrained Optimization}
\vspace{-1.5em}
$$ \max_{\{c_1,c_2\}} \quad U(c_1, c_2) = \ln c_1 + \beta \ln c_2 \quad s.t. \quad c_1 + \frac{1}{1+r} c_2 = m $$
 \begin{tikzpicture}[remember picture,overlay]
    \draw[gray!50, opacity=0.75] (0,-5.25) grid[step=0.75] (14.25,0);
  \end{tikzpicture}
\end{frame}


%%%%%%%%%%%%%%%%%%%%
\begin{frame}{Envelope Theorem with Constraints}
Value function:
$$
V=f\left(x^{*}(\alpha), y^{*}(\alpha), \alpha\right)
$$

\vspace{0.5em}
By envelope theorem:
$$
\frac{d V}{d \alpha}=\frac{\partial L^{*}}{\partial \alpha}
$$

\vspace{2em}
How does the optimal utility change due to changes in $r$, $y_1$, $y_2$, or $\beta$?
\end{frame}

%%%%%%%%%%%%%%%%%%%%
\begin{frame}{Interpretation of the Lagrange Multiplier}
Lagrangian function:
$$
L=f(x, y )+\lambda [c-g(x, y)]
$$
Substituting the solutions into the objective function, we get
$$
V = f(x^{*}(c), y^{*}(c))
$$
By the envelope theorem,
$$
\frac{d V}{d c}=\frac{\partial L^{*}}{\partial c}=\lambda^*
$$
\end{frame}

%%%%%%%%%%%%%%%%%%%%
\begin{frame}{Global Optimizers with Constraints}

Consider the problem:\\~\\
Maximize $f(x_1,x_2,...,x_n)$ subject to $g(x_1, x_2,...,x_n)=k$.\\~\\

The stationary point $(x_1^*, x_2^*, ..., x_n^*)$ of the lagrangian is a global maximum if:
\begin{enumerate}
  \item $f(x_1,x_2,...,x_n)$ is quasiconcave
  \item The constraint set is convex
\end{enumerate}
\end{frame}

%%%%%%%%%%%%%%%%%%%%
\begin{frame}{\vspace{1em} \huge A Few Last Words}
\vspace{0.7em}
\large \centering
Please fill the SOQs :) \\~\\
Thanks for a great
semester. \\~\\
Good luck and don’t be a stranger!
\end{frame}

\end{document}