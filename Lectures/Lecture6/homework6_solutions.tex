\documentclass{./../../Latex/homework}
\begin{document}
\thispagestyle{plain}
\myheader{Homework 6 Solutions}

%%%%%%%%%%%%%%%% Exercise 10.5
\subsection*{Exercise 10.5} 

\begin{enumerate}

%%%%%%%%% Question 1
\item[1.] 
\begin{enumerate}
\item $\begin{aligned}2 e^{2 t+4}\end{aligned}$
\item $\begin{aligned}-9 e^{1-9 t}\end{aligned}$
\item $\begin{aligned}2 t e^{t^2+1}\end{aligned}$
\item $\begin{aligned}-10 t e^{2-t^2}\end{aligned}$
\item $\begin{aligned}(2 a x+b) e^{a x^2+b x+c}\end{aligned}$
\item $\begin{aligned}\frac{d y}{d x}=x \frac{d}{d x} e^x+e^x \frac{d x}{d x}=x e^x+e^x=(x+1) e^x\end{aligned}$
\item $\begin{aligned}\frac{d y}{d x}=x^2\left(2 e^{2 x}\right)+2 x e^{2 x}=2 x(x+1) e^{2 x}\end{aligned}$
\item $\begin{aligned}\frac{d y}{d x}=a\left(x b e^{b x+c}+e^{b x+c}\right)=a(b x+1) e^{b x+c}\end{aligned}$ \\~\\
\end{enumerate}

%%%%%%%%% Question 3
\item[3.]  
\begin{enumerate}
\item $\begin{aligned} \frac{d y}{d t}=\frac{35 t^4}{7 t^5}=\frac{5}{t} \end{aligned}$
\item $\begin{aligned}\frac{d y}{d t}=\frac{a c t^{o-1}}{a t^c}=\frac{c}{t}\end{aligned}$
\item $\begin{aligned}\frac{d y}{d t}=\frac{1}{t+19}\end{aligned}$
\item $\begin{aligned}\frac{d y}{d t}=5 \frac{2(t+1)}{(t+1)^2}=\frac{10}{t+1}\end{aligned}$
\item $\begin{aligned}\frac{d y}{d x}=\frac{1}{x}-\frac{1}{1+x}=\frac{1}{x(1+x)}\end{aligned}$
\item $\begin{aligned}\frac{d y}{d x}=\frac{d}{d x}[\ln x+8 \ln (1-x)]=\frac{1}{x}+\frac{-8}{1-x}=\frac{1-9 x}{x(1-x)}\end{aligned}$
\item $\begin{aligned}\frac{d y}{d x}=\frac{d}{d x}[\ln 2 x-\ln (1+x)]=\frac{2}{2 x}-\frac{1}{1+x}=\frac{1}{x(1+x)}\end{aligned}$
\item $\begin{aligned}\frac{d y}{d x}=5 x^4 \frac{2 x}{x^2}+20 x^3 \ln x^2=10 x^3\left(1+2 \ln x^2\right)=10 x^3(1+4 \ln x)\end{aligned}$ \\
\end{enumerate}

%%%%%%%%% Question 7
\item[7.]  
\begin{enumerate}
\item Since $\ln y=\ln 3 x-\ln (x+2)-\ln (x+4)$, we have 
$$\frac{1}{y} \frac{d y}{d x}=\frac{1}{x}-\frac{1}{x+2}-\frac{1}{x+4}=\frac{8-x^2}{x(x+2)(x+4)}$$ 
Hence,
 $$\frac{d y}{d x}=\frac{8-x^2}{x(x+2)(x+4)} \cdot \frac{3 x}{(x+2)(x+4)}=\frac{3\left(8-x^2\right)}{(x+2)^2(x+4)^2}$$
 
\item Since $\ln y=\ln \left(x^2+3\right)+x^2+1$, we have $$\frac{1}{y} \frac{d y}{d x}=\frac{2 x}{x^2+3}+2 x=\frac{2 x\left(x^2+4\right)}{x^2+3}$$ 
Hence, $$\frac{d y}{d x}=\frac{2 x\left(x^2+4\right)}{x^2+3}\left(x^2+3\right) e^{x^2+1}=2 x\left(x^2+4\right) e^{x^2+1}$$ \\
\end{enumerate}
\end{enumerate}


%%%%%%%%%%%%%%%% Exercise 7.4
\subsection*{Exercise 7.4} 
\begin{enumerate}

%%%%%%%%% Question 1
\item[1.] 
\begin{enumerate}
\item[(a)]  $y=2 x_{1}^{3}-11 x_{1}^{2} x_{2}+3 x_{2}^{2} $
$$
\frac{d y}{d x_{1}}=6 x_{1}^{2}-22 x_{1} x_{2}, \quad  \quad 
\frac{d y}{d x_{2}}=-11 x_{1}^{2}+6 x_{2} $$ 
\item[(d)]  $ y=\dfrac{5 x_{1}+3}{x_{2}-2}$ \\
$$
\begin{aligned} \frac{d y}{d x_{1}} &=\frac{5\left(x_{2}-2\right)-\left(5 x_{1}+3\right) \cdot 0}{\left(x_{2}-2\right)^{2}}=\frac{5}{x_{2}-2} \\
\frac{d y}{d x_{2}} & = \frac{0 \cdot (x_2-2)-1 \cdot (5x_1+3)}{(x_{2}-2)^2}= \frac{ -(5x_1+3)}{(x_{2}-2)^2} \\
\end{aligned}
$$ \\
\end{enumerate}

%%%%%%%%% Question 2 & 3
\item[2.\& 3.] 
\begin{enumerate}
\item $f(x, y)=x^{2}+5 x y-y^{3}$
$$
\begin{aligned}
&f_{x}=2 x+5 y & \rightarrow f_{x}(1,2)=12 \\
&f_{y}=5 x-3 y^{2} & \rightarrow f_{y}(1,2)=-7
\end{aligned}
$$ 
\item $f(x, y)=\left(x^{2}-3 y\right)(x-2)$
$$
\begin{aligned}
f_{x} &=(2 x)(x-2)+\left(x^{2}-3 y\right) \cdot 1 \\
&=2 x^{2}-4 x+x^{2}-3 y=3 x^{2}-4 x-3 y \\
\end{aligned}
$$
Then $f_{x}(1,2)=3-4-6=-7$
$$
\begin{aligned}
&f_{y}=-3(x-2)+\left(x^{2}-3 y\right) \cdot 0=-3 x+6 \\
\end{aligned}
$$
Then $f_{y}(1,2)=-3+6=3$. \\
\end{enumerate}

%%%%%%%%% Question 5
\item[5.] $U=U\left(x_{1}, x_{2}\right)=\left(x_{1}+2\right)^{2}\left(x_{2}+3\right)^{3}$ \\
\begin{enumerate}
\item $$ U_{1}\left(x_{1}, x_{2}\right)=2\left(x_{1}+2\right)\left(x_{2}+3\right)^{3}=\frac{2 U}{x_{1}+2}$$
$$
U_{2}\left(x_{1}, x_{2}\right)=3\left(x_{1}+2\right)^{2}\left(x_{2}+3\right)^{2}=\frac{3 U}{x_{2}+3}
$$

\item $
\begin{aligned}
& U_{1}(3,3)=2(3+2)(3+3)^{3} = 2 \times 5 \times 6^{3}=2160
\end{aligned}
$ \\
\end{enumerate}

%%%%%%%%% Question 7
\item[7.] \begin{enumerate}
\item
$$
\begin{aligned}
&f(x, y, z)=x^{2}+y^{3}+z^{4} \\
&f_{x}=2 x \\
&f_{y}=3 y^{2} \\
&f_{z}=4 z^{3} \\
&\nabla f(x, y, z)=\left[\begin{array}{l}
2 x \\
3 y^{2} \\
4 z^{3}
\end{array}\right]
\end{aligned}
$$
\item
$$
\begin{aligned}
&f(x, y, z)=x y z \\
&f_{x}=y z \\
&f_{y}=xz \\
&f_{z}=x y \\
&\nabla f(x, y, z)=\left[\begin{array}{l}
yz \\
x z \\
x y
\end{array}\right]
\end{aligned}
$$
\end{enumerate}
\end{enumerate}

%%%%%%%%%%%%%%% Exercise 8.1
\subsection*{Exercise 8.1} 

\begin{enumerate}

%%%%%%%%% Question 1
\item[1.] \begin{enumerate}
\item[(a)] $y=-x^{3}-3 x$
$$d y=\left(-3 x^{2}-3\right) d x=-3\left(x^{2}+1\right) d x$$ \\
\end{enumerate}

%%%%%%%%% Question 4
\item[4.] $Q=k p^{-n}, k>0, n>0$ \\
\begin{enumerate}
\item $$\frac{d Q}{d p}=-n k p^{-n-1}$$
$$\varepsilon_{d}=\frac{d Q}{d p} \cdot \frac{p}{Q}=\frac{-n k p^{-n-1} \cdot p}{k p^{-n}}=-n$$
No, the elasticity does not depend on the price. 
\item When $n=1, \quad Q=\frac{k}{p}, \varepsilon_{d}=-1$
\end{enumerate}
\begin{tikzpicture}
\begin{axis}[axis lines = left, xlabel = \(p\), ylabel = {\(Q\)},
	xmin = 0, xmax = 10, ymin=0, xtick={11}, ytick={-1}
	]
\addplot [domain=0.5:9.5, samples=100, color=red, line width=1.25]
{100/x};
\end{axis}
\end{tikzpicture}


%%%%%%%%% Question 6
\item[6.] $\quad Q=100-2 P+0.02 Y$
$$
P=20, Y=5000 \longrightarrow Q=100-40+100=160
$$
\begin{enumerate}
\item $$\frac{d Q}{d P} \cdot \frac{P}{Q}=-2 \cdot \frac{20}{160}=-0.25$$ 
\item $$\frac{d Q}{d Y} \cdot \frac{Y}{Q}=0.02 \cdot \frac{5000}{160}=0.625$$ \\
\end{enumerate}
\end{enumerate}

%%%%%%%%%%%%%%% Exercise 8.2
\subsection*{Exercise 8.2} 
\begin{enumerate}

%%%%%%%%% Question 3
\item[3.]
\begin{enumerate}
\item[(a)] $$y=\frac{x_{1}}{x_{1}+x_{2}}$$ \\
$$
\frac{d y}{d x_{1}}=\frac{1\left(x_{1}+x_{2}\right)-1 \cdot x_{1}}{\left(x_{1}+x_{2}\right)^{2}}=\frac{x_{2}}{\left(x_{1}+x_{2}\right)^{2}}
$$ \\
$$
\begin{aligned}
& \frac{d y}{d x_{2}}=\frac{0 \cdot\left(x_{1}+x_{2}\right)-1 x_{1}}{\left(x_{1}+x_{2}\right)^{2}}=\frac{-x_{1}}{\left(x_{1}+x_{2}\right)^{2}}
\end{aligned}
$$ \\
$$
d y=\frac{x_{2}}{\left(x_{1}+x_{2}\right)^{2}} \cdot d x_{1}-\frac{x_{1}}{\left(x_{1}+x_{2}\right)^{2}} d x_{2}
$$ \\
\end{enumerate}

%%%%%%%%% Question 4
\item[4.] $Q=a+b P^{2}+R^{1 / 2} \quad(a<0, b>0)$ \\
$$
\varepsilon_{Q P}=\frac{d Q}{d P} \cdot \frac{P}{Q}=\frac{2 b P \cdot P}{Q}=\frac{2 b P^{2}}{a+b P^{2}+R^{1 / 2}}
$$ \\
$$
\varepsilon_{Q R}=\frac{d Q}{d R} \cdot \frac{R}{Q}=\frac{1}{2} R^{\frac{1}{2}-1} \cdot \frac{R}{Q}=\frac{R^{1 / 2}}{2\left(a+b P^{2}+R^{1 / 2}\right)}
$$ \\

%%%%%%%%% Question 5
\item[5.] $$\quad \frac{d \varepsilon_{Q P}}{d P}=\frac{4 b P\left(a+b P^{2}+R^{1 / 2}\right)-2 b P\left(2 b P^{2}\right)}{\left(a+b P^{2}+R^{1 / 2}\right)^{2}}$$
$$
=\frac{4 b P\left(a+R^{1 / 2}\right)}{\left(a+b P^{2}+R^{1 / 2}\right)^{2}}
$$
Denominator is positive. Numerator is positive when $a+R^{1 / 2}>0$.
So $\frac{d \varepsilon_{Q P}}{d P} \geq 0$ when $ a+R^{1 / 2} \geq 0$ and $\frac{d \varepsilon_{Q P}}{d P} < 0$ when $ a+R^{1 / 2} < 0$. \\
$$
\begin{aligned}
&\frac{d \varepsilon_{Q R}}{d R}=\frac{-b P^{2} R^{-1 / 2}}{\left(a+b P^{2}+R^{1 / 2}\right)^{2}}<0 \\~\\
&\frac{d \varepsilon_{Q R}}{d P}=\frac{-b P R^{-1 / 2}}{\left(a+b P^{2}+R^{1 / 2}\right)^{2}}<0
\end{aligned}
$$ \\
$$
\begin{aligned}
\frac{d \varepsilon_{Q, R}}{d R} &=\frac{R^{-1 / 2}\left(a+b P^{2}+R^{1 / 2}\right)-R^{-1 / 2} R^{1 / 2}}{4\left(a+b P^{2}+R^{1 / 2}\right)^{2}} \\
&=\frac{R^{-1 / 2}\left(a+b P^{2}\right)}{4\left(a+b P^{2}+R^{1 / 2}\right)^{2}}
\end{aligned}
$$
Similar reasoning as before, $\frac{d \varepsilon_{Q R}}{d R} \geq 0$ if $a+b p^{2} \geq 0$ and 
$\frac{d \varepsilon_{Q R}}{d R} < 0$ if $a+b p^{2} < 0$. \\~\\

%%%%%%%%% Question 6
\item[6.] $\quad x=y_{f}^{1 / 2}+p^{-2}$
$$
\varepsilon_{x p}=\frac{d x}{d p} \cdot \frac{p}{x}=\frac{-2 p^{-2}}{y_{f}^{1 / 2}+p^{-2}}=\frac{-2}{y_{f}^{1 / 2} p^{2}+1}<0
$$

%%%%%%%%% Question 7
\item[7.] 
\begin{enumerate}

\item[(a)] $U=7 x^{2} y^{3}$
$$
\begin{aligned}
d u &=u_{x} d x+u_{y} \cdot d y \\
&=14 x y^{3} \cdot d x+21 x^{2} y^{2} \cdot d y \\
&=7 x y^{2}(2 y \cdot d x+3 x \cdot d y)
\end{aligned}
$$

\item[(f)] $U =(x-3 y)^{3}$
$$
\begin{aligned}
d u &=3(x-3 y)^{2} d x-3(x-3 y)^{2}(-3) \\
&=3(x-3 y)^{2}(d x-3 d y)
\end{aligned}
$$ \\
\end{enumerate}
\end{enumerate}

%%%%%%%%%%%%%%% Exercise 8.4
\subsection*{Exercise 8.4} 

\begin{enumerate}
%%%%%%%%% Question 2
\item[2.] 
\begin{enumerate}
\item
$$
\begin{aligned}
z &=f(x, y)=x^{2}-8 x y-y^{3} \\
x &=3 t \\
y &=1-t \\~\\
\frac{d z}{d t} &=f_x \frac{d x}{d t}+f_y \cdot \frac{d y}{d t} \\
&=(2 x-8 y) 3+\left(-8 x-3 y^{2}\right)(-1) \\
&=6 x-24 y+8 x+3 y^{2} \\
&=42 t-24(1-t)+3(1-t)^{2} \\
&=3 t^{2}+60 t-21
\end{aligned}
$$

\item
$$
\begin{aligned}
z &=f(u, v, t)=7 u+v t \\
u &=2 t^{2}, v=t+1 \\~\\
\frac{d z}{d t} &=f_u \frac{d u}{d t}+f_v \cdot \frac{d v}{d t}+f_t \frac{d t}{d t} \\
&=7(4 t)+t .1+v \\
&=28 t+t+t+1 \\
&=30 t+1
\end{aligned}
$$

\item
$$
\begin{aligned}
z &=f(x, y, t) \\
x &=a+b t \\
y &=c+R t \\~\\
\frac{d z}{d t} &=f_{x} \frac{d x}{d t}+f_{y} \frac{d y}{d t}+f_{t} \\
&=b f_{x}+R f_{y}+f_{t}
\end{aligned}
$$
\end{enumerate}

%%%%%%%%% Question 4
\item[4.] \begin{enumerate}
\item
$$
\begin{aligned}
w &=f(x, y, u)=a x^{2}+b x y+c u \\
x &=\alpha u+\beta v \\
y &=\gamma u \\~\\
\frac{d w}{d u} &=f x \frac{d x}{d u}+f y \cdot \frac{d y}{d u}+f_{u} \\
&=(2 a x+b y) \alpha+\gamma b x+c \\
&=(2 a+\gamma b) x+\alpha b y+c \\
&=(2 a+\gamma b)(\alpha u+\beta v)+\gamma \alpha b u+c
\end{aligned}
$$
\item
$$
\begin{aligned}
&w=f\left(x_{1}, x_{2}\right) \\
&x_{1}=5 u^{2}+3 v \\
&x_{2}=u-4 v^{3}
\end{aligned}
$$ \\
$$
\begin{aligned}
\frac{d w}{d u} &=f_{1} \frac{d x_{1}}{d u}+f_{2} \frac{d x_{2}}{d u} \\
&=f_{1} \cdot 10 u+f_{2}=10 u f_{1}+f_{2} \\~\\
\frac{d w}{d v} &=f_{1} \frac{d x_{1}}{d v}+f_{2} \frac{d x_{2}}{d v} \\
&=f_{1} \cdot 3+f_{2}\left(-12 v^{2}\right)=3 f_{1}-12 v^{2} f_{2}
\end{aligned}
$$ \\
\end{enumerate}
\end{enumerate}

\end{document}