\documentclass{./../../Latex/homework}
\begin{document}
\thispagestyle{plain}
\myheader{Homework 4 Solutions}

%%%%%%%%%%%%%%%% Exercise 5.3
\subsection*{Exercise 5.3}

% Question 1
\begin{enumerate}
\item 
$$
\begin{aligned}
\left|\begin{array}{ccc}
4 & 0 & -1 \\
2 & 1 & -7 \\
3 & 3 & 9
\end{array}\right| &=4\left|\begin{array}{rr}
1 & -7 \\
3 & 9
\end{array}\right|-0\left|\begin{array}{cc}
2 & -7 \\
3 & 9
\end{array}\right|-1\left|\begin{array}{ll}
2 & 1 \\
3 & 3
\end{array}\right| \\
&=4(9+21)-0(18+21)-1(6-3) \\
&=120-0-3=117
\end{aligned}
$$

Interchanging rows and columns:\\

\( \begin{aligned}\left|\begin{array}{ccc}4 & 2 & 3 \\ 0 & 1 & 3 \\ -1 & -7 & 9\end{array}\right| &=4\left|\begin{array}{cc}1 & 3 \\ -7 & 9\end{array}\right|-2\left|\begin{array}{cc}0 & 3 \\ -1 & 9\end{array}\right|+3\left|\begin{array}{cc}0 & 1 \\ -1 & -7\end{array}\right| \\ &=4(9+21)-2(0+3)+3(0+1) \\ &=120-6+3=117 \end{aligned} \) \\

Interchange row 1 and 2:
$$
\begin{aligned}
\left|\begin{array}{rrr}
2 & 1 & -7 \\
4 & 0 & -1 \\
3 & 3 & 9
\end{array}\right| &=2\left|\begin{array}{rr}
0 & -1 \\
3 & 9
\end{array}\right|-1\left|\begin{array}{cc}
4 & -1 \\
3 & 9
\end{array}\right|-7\left|\begin{array}{ll}
4 & 0 \\
3 & 3
\end{array}\right| \\
&=2(0+3)-1(36+3)-7(12-0) \\
&=6-39-84=-117
\end{aligned}
$$
You can verify the other two properties in the same way. \\~\\

% Question 5
\item[5.] 
\begin{enumerate}
\item[(a)] $$
\begin{array}{l}
4\left|\begin{array}{cc}
1 & -3 \\
1 & 0
\end{array}\right|+1\left|\begin{array}{cc}
19 & 1 \\
7 & 1
\end{array}\right| \\
=4(0+3)+1(19-7) \\
=\quad 12+12=24
\end{array}
$$

Non-singular, rank = 3. \\

\item[(b)] $$
\begin{aligned}
& 5\left|\begin{array}{cc}
-2 & 1 \\
0 & 3
\end{array}\right|+6 \left| \begin{array}{ll}
4 & 1 \\
7 & 3
\end{array}\right| \\
=& 5(-6-0)+6(12-7) \\
=&-30+30=0
\end{aligned}
$$

Singular. Rank<3, will need to put in echelon form to find exact rank. \\

\item[(c)]
\[
\begin{aligned} & 7\left|\begin{array}{cc}1 & 4 \\ -3 & -4\end{array}\right|+1\left|\begin{array}{cc}1 & 4 \\ 13 & -4\end{array}\right| \\=& 7(-4+12)+1(-4-52) \\=& 56-56=0 \end{aligned}
\]

Singular. Rank <3, will need to put in echelon form to find exact rank. \\

\item[(d)]
$$ \begin{aligned} &-3\left|\begin{array}{ll}9 & 5 \\ 8 & 6\end{array}\right|-1\left|\begin{array}{cc}-4 & 9 \\ 10 & 8\end{array}\right| \\
 =&-3(54-40)-1(-32-90) \\
 =&-42+122=80 \end{aligned}$$
Non-singular, rank = 3. \\~\\
\end{enumerate}

% Question 8
\item[8.] 
\begin{enumerate}
\item False. While we can find the transpose of any matrix, the determinant is only defined for square matrices. 
\item False. Multiplying each element of an $n \times n$ by 2 will increase the determinant by $2^n$ times. 
\item $A$ cannot be 0, but the determinant of $A$ can. In any case, if $|A|$=0, $A$ is singular.
\end{enumerate}

\end{enumerate}

%%%%%%%%%%%%%%%% Exercise 5.4
\subsection*{Exercise 5.4} 

\begin{enumerate}

% Question 2
\item[2.] Note that $$
A^{-1}=\frac{1}{|A|} \operatorname{adj} A
$$

Here, \( \operatorname{adj} A=C^{\prime} \) where \( C=\left[\left|C_{i j}\right|\right] \).

\begin{enumerate}

\item 
$$ |A|=5-0=5 $$
$$\operatorname{adj} A=\left[\begin{array}{ll}\left|C_{11}\right| & \left|C_{21}\right| \\ \left|C_{12}\right| & \left|C_{22}\right|\end{array}\right]=\left[\begin{array}{cc}1 & -2 \\ 0 & 5\end{array}\right] $$
$$A^{-1}=\frac{1}{|A|} \operatorname{adj} A=\frac{1}{5}\left[\begin{array}{cc}1 & -2 \\ 0 & 5\end{array}\right]=\left[\begin{array}{cc}1 / 5 & -2 / 5 \\ 0 & 1\end{array}\right] $$

\item $$
\begin{array}{l}
|B|=-2-0=-2 \\
B^{-1}=\frac{-1}{2}\left[\begin{array}{rr}
2 & 0 \\
-9 & -1
\end{array}\right]
\end{array} 
$$

\item $$
\begin{array}{l}
|C|=-3-21=-24 \\
C^{-1}=\frac{-1}{24}\left[\begin{array}{cc}
-1 & -7 \\
-3 & 3
\end{array}\right]
\end{array}
$$

\item $$
\begin{array}{l}
|D|=21-0=21 \\
D^{-1}=\frac{1}{21}\left[\begin{array}{rr}
3 & -6 \\
0 & 7
\end{array}\right]
\end{array}
$$
\end{enumerate}

% Question 3
\item[3.] 
\begin{enumerate}
\item Step 1: Exchange the two diagonal elements.\\
Step 2: Multiply both the off-diagonal elements by -1.

\item Step 3: Multiply the resulting matrix from steps 1 and 2 by \(1/|A|\).
\end{enumerate}

% Question 4
\item[4.] (a)$$
\begin{aligned}
|E| &=1\left|\begin{array}{ll}
7 & 3 \\
2 & 0
\end{array}\right|+1\left|\begin{array}{rr}
4 & -2 \\
7 & 3
\end{array}\right| \\
&=1(0-6)+1(12+14) \\
&=-6+26=20
\end{aligned}
$$

$$
\begin{aligned}
\operatorname{adj} E &=\left[\begin{array}{ccc}
\left|C_{11}\right| & \left|C_{21}\right| & \left|C_{31}\right| \\
\left|C_{12}\right| & \left|C_{22}\right| & \left|C_{32}\right| \\
\left|C_{13}\right| & \left|C_{23}\right| & \left|C_{33}\right|
\end{array}\right] \\
&=\left[\begin{array}{ccc}
3 & 2 & -3 \\
-7 & 2 & 7 \\
-6 & -4 & 26
\end{array}\right]
\end{aligned}
$$
$$ E^{-1}=\frac{1}{|E|} a d j E $$ 

(b)
$$
F^{-1}=\frac{-1}{10}\left[\begin{array}{ccc}
0 & 2 & -3 \\
10 & -6 & -1 \\
0 & -4 & 1
\end{array}\right]
$$

(c)
$$ \left|C_{11}\right|=\left|\begin{array}{ll}0 & 1 \\ 1 & 0\end{array}\right|=-1 
\quad  \left|C_{12}\right|=-\left|\begin{array}{ll}0 & 1 \\ 0 & 0\end{array}\right|=0 
\quad  \left|C_{13}\right|=\left|\begin{array}{ll}0 & 0 \\ 0 & 1\end{array}\right|=0 
$$
$$ \left|C_{21}\right|=-\left|\begin{array}{ll}0 & 0 \\ 1 & 0\end{array}\right|=0 
\quad  \left|C_{22}\right|=\left|\begin{array}{ll}0 & 1 \\ 0 & 0\end{array}\right|=0 
\quad  \left|C_{23}\right|=-\left|\begin{array}{ll}1 & 0 \\ 0 & 1\end{array}\right|=-1 $$
$$
\left|C_{31}\right|=\left|\begin{array}{cc}
0 & 0 \\
0 & 1
\end{array}\right|=0 \quad\left|C_{32}\right|=-\left|\begin{array}{cc}
1 & 0 \\
0 & 1
\end{array}\right|=-1 \quad\left|C_{33}\right|=\left|\begin{array}{ll}
1 & 0 \\
0 & 0
\end{array}\right|=0
$$
$$ G^{-1}=-1\left[\begin{array}{ccc}-1 & 0 & 0 \\ 0 & 0 & -1 \\ 0 & -1 & 0\end{array}\right]=\left[\begin{array}{lll}1 & 0 & 0 \\ 0 & 0 & 1 \\ 0 & 1 & 0\end{array}\right] $$ \\

(d) $$ H^{-1}=\left[\begin{array}{lll}1 & 0 & 0 \\ 0 & 1 & 0 \\ 0 & 0 & 1\end{array}\right] $$ \\


% Question 6
\item[6.] (a) $$
\begin{array}{l}
A=\left[\begin{array}{ll}
4 & 3 \\
2 & 5
\end{array}\right] \quad v=\left[\begin{array}{l}
x \\
y
\end{array}\right] \quad b=\left[\begin{array}{ll}
28 \\
42
\end{array}\right] \\~\\
\because A v=b \Rightarrow A^{-1} A v=A^{-1} b \Rightarrow v=A^{-1}b
\end{array}
$$

\(|A| =20-6=14 \)

\(A^{-1} =\frac{1}{14}\left[\begin{array}{cc}5 & -3 \\ -2 & 4\end{array}\right] \)
\[
\begin{aligned}v=A^{-1} b &=\frac{1}{14}\left[\begin{array}{cc}5 & -3 \\ -2 & 4\end{array}\right]\left[\begin{array}{l}28 \\ 42\end{array}\right] \\ &=\frac{1}{14}\left[\begin{array}{l}140-126 \\ -56+168\end{array}\right] = \frac{1}{14}\left[\begin{array}{l}14 \\ 112\end{array}\right] =\left[\begin{array}{l}1 \\ 8\end{array}\right] \end{aligned}
\]\\
So \(x=1\) and \(y=8\). \\~\\

(b)
$$ A=\left[\begin{array}{ccc}4 & 1 & -5 \\ -2 & 3 & 1 \\ 3 & -1 & 4\end{array}\right] \quad x=\left[\begin{array}{l}x_{1} \\ x_{2} \\ x_{3}\end{array}\right] \quad b=\left[\begin{array}{c}8 \\ 12 \\ 5\end{array}\right] $$ 

$$
\begin{aligned}
|A| &=4\left|\begin{array}{cc}
3 & 1 \\
-1 & 4
\end{array}\right|-1\left|\begin{array}{cc}
-2 & 1 \\
3 & 4
\end{array}\right|-5\left|\begin{array}{cc}
-2 & 3 \\
3 & -1
\end{array}\right| \\
&=4(12+1)-1(-8-3)-5(2-9) \\
&=52+11+35=98
\end{aligned}
$$
$$
\operatorname{adj} A =\left[\begin{array}{ccc}
13 & 1 & 16 \\
11 & 31 & 6 \\
-7 & 7 & 14
\end{array}\right] $$
$$
x=A^{-1} b =\frac{1}{98}\left[\begin{array}{ccc}
13 & 1 & 16 \\
11 & 31 & 6 \\
-7 & 7 & 14
\end{array}\right]_{3 \times 3} \left[\begin{array}{l}
8 \\
12 \\
5
\end{array}\right]_{3 \times 1} \\
=\frac{1}{98}\left[\begin{array}{c}
196 \\
490 \\
98
\end{array}\right]_{3 \times 1}=\left[\begin{array}{c}
2 \\
5 \\
1
\end{array}\right]
$$ \\

% Question 7
\item[7.] If a matrix is its own inverse, we would need that \(A^2 = I \). This is true for the identity matrix. There are other possibilities such as matrix \(G\) in exercise 4. \\~\\
\end{enumerate}

%%%%%%%%%%%%%%%% Exercise 5.5
\subsection*{Exercise 5.5} 

\begin{enumerate}
% Question 1 & 2
\item[1 \& 2.]  
(a) $A=\left[\begin{array}{cc}3 & -2 \\ 2 & 1\end{array}\right] \rightarrow|A|=7 \quad b=\left[\begin{array}{c}6 \\ 11\end{array}\right]$ \\

By Cramer's Rule:

$$
\begin{aligned}
A_{1} & =\left[\begin{array}{rr}
6 & -2 \\
11 & 1
\end{array}\right] \rightarrow\left|A_{1}\right|=28 \\
A_{2} & =\left[\begin{array}{ll}
3 & 6 \\
2 & 11
\end{array}\right] \rightarrow\left|A_{2}\right|=21 \\
x_{1}^{*} & =\frac{\left|A_{1}\right|}{|A|}=\frac{28}{7}=4 \\
x_{2}^{*} & =\frac{\left|A_{2}\right|}{|A|}=\frac{21}{7}=3
\end{aligned}
$$

Taking the inverse:

$$
\begin{aligned}
& A^{-1}=\frac{1}{|A|} \operatorname{adj}=\frac{1}{7}\left[\begin{array}{cc}1 & 2 \\-2 & 3\end{array}\right] \\
& \left[\begin{array}{l}x_{1}^{*} \\x_{2}^{*}\end{array}\right]=A^{-1} b=\frac{1}{7}\left[\begin{array}{rr}1 & 2 \\-2 & 3\end{array}\right]\left[\begin{array}{l}6 \\11\end{array}\right]=\frac{1}{7}\left[\begin{array}{l}28 \\21\end{array}\right]=\left[\begin{array}{l}4 \\3\end{array}\right] 
\end{aligned}
$$

(b)
$$
A=\left[\begin{array}{cc}
-1 & 3 \\
4 & -1
\end{array}\right] \rightarrow|A|=-11 \quad b=\left[\begin{array}{c}
-3 \\
12
\end{array}\right]
$$

By Cramer's rule:

$$
\begin{aligned}
& \left|A_{1}\right|=\left|\begin{array}{cc}
-3 & 3 \\
12 & -1
\end{array}\right|=-33 \rightarrow x_{1}^{*}=\frac{-33}{-11}=3 \\
& \left|A_{2}\right|=\left|\begin{array}{cc}
-1 & -3 \\
4 & 12
\end{array}\right|=0 \rightarrow x_{2}^{*}=\frac{0}{-11}=0
\end{aligned}
$$

By taking the inverse:

$$
\left[\begin{array}{l}
x_{1}^{*} \\
x_{2}
\end{array}\right]=\frac{-1}{11}\left[\begin{array}{cc}
-1 & -3 \\
-4 & -1
\end{array}\right]\left[\begin{array}{c}
-3 \\
12
\end{array}\right]=\frac{-1}{11}\left[\begin{array}{c}
-33 \\
0
\end{array}\right]=\left[\begin{array}{l}
3 \\
0
\end{array}\right] 
$$

(c)
$$
\begin{aligned}
& |A|=\left|\begin{array}{cc}
8 & -7 \\
1 & 1
\end{array}\right|=15 \\
& \left|A_{1}\right|=\left|\begin{array}{cc}
9 & -7 \\
3 & 1
\end{array}\right|=30 \quad \rightarrow x_{1}^{*}=\frac{30}{15}=2 \\
& \left|A_{2}\right|=\left|\begin{array}{cc}
8 & 9 \\
1 & 3
\end{array}\right|=15 \rightarrow x_{2}^{*}=\frac{15}{15}=1 \\
\end{aligned}
$$
Alternatively, 
$$\left[\begin{array}{c}
x_{1}^{*} \\
x_{2}^{*}
\end{array}\right]=\frac{1}{15}\left[\begin{array}{cc}
1 & 7 \\
-1 & 8
\end{array}\right]\left[\begin{array}{l}
9 \\
3
\end{array}\right]=\left[\begin{array}{l}
2 \\
1
\end{array}\right]
$$

(d)
$$
\begin{aligned}
& |A|=\left|\begin{array}{cc}
5 & 9 \\
7 & -3
\end{array}\right|=-78 \\
& \left|A_{1}\right|=\left|\begin{array}{cc}
14 & 9 \\
4 & -3
\end{array}\right|=-78 \rightarrow x_{1}^{*}=1 \\
& \left|A_{2}\right|=\left|\begin{array}{cc}
5 & 14 \\
7 & 4
\end{array}\right|=-78 \rightarrow x_{2}^{*}=1
\end{aligned}
$$

Alternatively, $$\left[\begin{array}{l}x_{1}^{2} \\ x_{2}^{*}\end{array}\right]=\frac{-1}{78}\left[\begin{array}{cc}-3 & -9 \\ -7 & 5\end{array}\right]\left[\begin{array}{c}14 \\ 4\end{array}\right]=\left[\begin{array}{l}1 \\ 1\end{array}\right]$$

% Question 3
\item[3.] (a) $$
\begin{aligned}
|A| & =\left|\begin{array}{rrr}
8 & -1 & 0 \\
0 & 2 & 5 \\
2 & 0 & 3
\end{array}\right| \\
& =8\left|\begin{array}{cc}
2 & 5 \\
0 & 3
\end{array}\right|-1\left|\begin{array}{ll}
0 & 5 \\
2 & 3
\end{array}\right|=48-10=38
\end{aligned}
$$

$$
\begin{aligned}
\left|A_{1}\right|= & \left|\begin{array}{ccc}
16 & -1 & 0 \\
5 & 2 & 5 \\
7 & 0 & 3
\end{array}\right| \\
& 16\left|\begin{array}{cc}
2 & 5 \\
0 & 3
\end{array}\right|-(-1)\left|\begin{array}{rr}
5 & 5 \\
7 & 3
\end{array}\right|=96-20=76 \\
& x_{1}^{*}=\frac{76}{38}=2
\end{aligned}
$$


$$
\begin{aligned}
\left|A_{2}\right| & =\left|\begin{array}{ccc}
8 & 16 & 0 \\
0 & 5 & 5 \\
2 & 7 & 3
\end{array}\right| \\
& =8\left|\begin{array}{cc}
5 & 5 \\
7 & 3
\end{array}\right|-16\left|\begin{array}{cc}
0 & 5 \\
2 & 3
\end{array}\right| \\
& =8 \times-20-16 \times-10=-160+160=0 \\
& x_{2}^{*}=0
\end{aligned}
$$

$$
\begin{aligned}
\left|A_{3}\right| & =\left|\begin{array}{ccc}
8 & -1 & 16 \\
0 & 2 & 5 \\
2 & 0 & 7
\end{array}\right| \\
& =8\left|\begin{array}{cc}
2 & 5 \\
0 & 7
\end{array}\right|+1\left|\begin{array}{cc}
0 & 5 \\
2 & 7
\end{array}\right|+16\left|\begin{array}{ll}
0 & 2 \\
2 & 0
\end{array}\right| \\
& =112-10-64=38 \rightarrow x_{3}^{*}=1 
\end{aligned}
$$

(d) 
$$
\begin{aligned}
|A| & =\left|\begin{array}{rrr}
-1 & 1 & 1 \\
1 & -1 & 1 \\
1 & 1 & -1
\end{array}\right| \\
& =1\left|\begin{array}{rr}
-1 & 1 \\
1 & -1
\end{array}\right|-1\left|\begin{array}{rr}
1 & 1 \\
1 & -1
\end{array}\right|+1\left|\begin{array}{rr}
1 & -1 \\
1 & 1
\end{array}\right| \\
& =0+2+2=4
\end{aligned}
$$

$$
\begin{aligned}
& \left|A_{1}\right|=\left|\begin{array}{ccc}a & 1 & 1 \\b & -1 & 1 \\c & 1 & -1\end{array}\right| \\
& =a\left|\begin{array}{cc}-1 & 1 \\1 & -1\end{array}\right|-1\left|\begin{array}{cc}b & 1 \\c & -1\end{array}\right|+1\left|\begin{array}{cc}b & -1 \\c & 1\end{array}\right| \\
& =0-(-b-c)+(b+c)=2(b+c) \\~\\
& \left|A_{2}\right|=\left|\begin{array}{rrr}-1 & a & 1 \\1 & b & 1 \\1 & c & -1\end{array}\right| \\
& =-1\left|\begin{array}{cc}b & 1 \\c & -1\end{array}\right|-a\left|\begin{array}{cc}1 & 1 \\1 & -1\end{array}\right|+1\left|\begin{array}{ll}1 & b \\1 & c\end{array}\right| \\
& =-1(-b-c)-a(-1-1)+1(c-b) \\
& =b+c-2 a+c-b=2(a+c) \\~\\
& \left|A_{3}\right|=\left|\begin{array}{rrr}-1 & 1 & a \\1 & -1 & b \\1 & 1 & c\end{array}\right| \\
& =-1\left|\begin{array}{rr}-1 & b \\1 & c\end{array}\right|-1\left|\begin{array}{ll}1 & b \\1 & c\end{array}\right|+a\left|\begin{array}{cc}1 & -1 \\1 & 1\end{array}\right| \\
& =-1(-c-b)-1(c-b)+2 a \\
& =c+b-c+b+2 a=2(a+b) \\
\end{aligned}
$$

$$
x^{*}=\frac{b+c}{2} \quad y^{*}=\frac{a+c}{2} \quad z^{*}=\frac{a+b}{2}
$$

\end{enumerate}


\end{document}