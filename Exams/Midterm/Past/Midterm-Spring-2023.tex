\documentclass{./../../../Latex/tests}
\begin{document}
\thispagestyle{plain}
\myheader{Spring 2023 Midterm Exam}
\rhead{Spring 2023 Midterm Exam}
\vspace{0.5em}
\testHeader{110}{30}

\begin{enumerate}	

%%%%%%%%%%%%%% Question 1
\item (8 pts) Answer the following questions (1 point each)
\begin{enumerate}
% Part 
\item Consider two sets $A$ and $B$, where $A$ is the set of all odd real numbers and $B$ is the set of all real numbers. What is the intersection of $A$ and $B$? 
\vspace{3cm}
\item Expand the following summation expression: $\sum_{i=0}^{3}(x+i)^{2}$
 \vspace{3cm}
\item Find the inverse of $f(x) = \frac{x-2}{3}$.
\vspace{3cm}
% Part 
\item Why do we need a matrix to be nonsingular when solving systems of linear equations?
\begin{itemize}
\item[$\square$] To ensure that the system of equations has a unique solution.
\item[$\square$] To ensure that the system of equations has no solutions.
\item[$\square$] To ensure that the system of equations has infinitely many solutions.
\item[$\square$] It does not matter if the matrix is singular or nonsingular.  \\~\\
 \end{itemize}
% Part
\item Is the following function continuous? Is it differentiable?
$$ f(x) = \begin{cases}
	4 \quad \text{if } x<2 \\
	10 \quad \text{if } x\geq2
\end{cases} $$
 \vspace{2cm}
\item For the function $f(x) = \ln x$, $f^{\prime} (x)=1/x$ 
  \begin{itemize}
  	\item[$\square$] True 
  	\item[$\square$] False \\~\\
  \end{itemize}  
  \item Find the derivative of $y = \frac{1}{x}$.
   \vspace{3cm}
  \item Find the derivative of $y = (2-3x)(1+x)$.
  \vspace{3cm}
\end{enumerate}

%%%%%%%%%%%%%% Question 2
\newpage
\item (10 pts) Consider the following system of equations:
\begin{align*}
4 x + 3y -2z & = 7 \\
x+y & = 5 \\
3x+z &= 4	
\end{align*}
\begin{enumerate}
  \item (1.5 pt) Write this system of equations in matrix format, i.e., $$ Av=b $$
  What is $A$, $v$, and $b$ equal to?
  \vspace{5.5cm}
   \item (3 pts) Calculate the adjoint of $A$. 
  \newpage
  \item (2 pts) Calculate the determinant of $A$. Is $A$ nonsingular?
  \vspace{6.5cm}
  \item (1.5 pt) If you premultiply $A^{-1}$ on both sides of the equation $ Av=b $, you should be able to derive an expression to solve for $v$. Write down this expression. 
  \vspace{3.5cm}
  \item (2 pts) Using the expression in $(d)$ solve for $v^*$. 
\end{enumerate}


\newpage
%%%%%%%%%%%%%% Question 3
\item (6 pts) Fun with Calculus!
\begin{enumerate}
\item (3 pts) Demand for a good as a function of its price is given as follows:
$$ Q(p) = p^{-\frac{1}{1+\alpha}}  $$
Calculate the elasticity of demand with respect to price. (Note: You can also take the log of both sides of the equation and write $\ln Q = -\frac{1}{1+\alpha} \cdot \ln p$, and use that equation if you like.)
\vspace{7cm}
\item (3 pts) Suppose that aggregate income $Y$ and population $P$ are given by:
$$Y(t) = \ln P(t), \quad \quad P(t) = ae^{rt}$$ 
where $c, a$, and $r$ are constants. $t$ denotes time. Find the growth rate of income, which is given by the derivative of $Y$ with respect to $t$.
\end{enumerate}

%%%%%%%%%%%%%% Question 4
\newpage
\item (6 pts) Consider the following production function with two inputs, capital ($K$) and labor ($L$):
$$ Q = 2K^{1/2}L^{1/2} $$
The marginal product of an input is given by the partial derivative of the production function with respect to that input variable. 
\begin{enumerate}
\item (3 pts) Show that the marginal product of capital (MPK) and labor (MPL)for the above production function are given by:
$$ MPK = \frac{1}{2}\cdot\frac{Q}{K} \quad \quad MPL = \frac{1}{2}\cdot \frac{Q}{L} $$
\vspace{4cm}
\item (2 pts) Now, say that in equilibrium, wages ($w$) are equal to the marginal product of labor i.e.
$$ w = \frac{1}{2}\cdot \frac{Q}{L} = K^{1/2}L^{-1/2}   $$
Given $K=100$, write labor demand $L$ as a function of wages $w$. (Essentially, you are finding the inverse of a function).
\vspace{4cm}
\item (1 pt) Given your answer in (b), do you think labor demand increases or decreases with an increase in wages?
\end{enumerate}

\end{enumerate}
\end{document}